\documentclass[12pt]{article}
\usepackage{amsmath}
\usepackage{amssymb}
\usepackage{bm}
\usepackage{mathtools}
\usepackage[a4paper, margin=1in]{geometry}
\usepackage{graphicx}
\usepackage{hyperref}
\usepackage[font=footnotesize]{caption}
\usepackage{fontspec}
\setmainfont{Arial}


\begin{document}

\title{Multiple stages of automaticity}
\author{}
% \date{\today}

\maketitle

\section*{Introduction}

We built the model shown in Fig. \ref{fig_model} and
simulated it in a minimal forced-choice category learning
task with just two stimuli, one per category. After each
response, the model received correct/incorrect feedback.
Dopamine-dependent, three-factor synaptic plasticity was
implemented at all cortical-subcortical synapses (VIS-DMS
and PM-DLS). Two-factor Hebbian learning governed all
cortical-cortical synapses (VIS-PM and PM-M1). Each
simulation ran for 2000 trials, with different simulations
exploring the effects of lesioning either the DMS or DLS at
various learning stages. Lesions were simulated by setting
the membrane potential of the affected neurons to zero. We
also froze synaptic weight updates after the lesion.
Preliminary results show:

\begin{itemize}

    \item Lesions early in learning -- from trial 100 -- to
        both the DMS and DLS prevent learning (Fig.
        \ref{fig_dms_100} and fig.  \ref{fig_dls_100}).

    \item Lesions at intermediate stages of learning -- from
        trial 500 -- to the DLS prevent learning (Fig.
        \ref{fig_dls_500}), but lesions to the DMS do not
        (Fig. \ref{fig_dms_500}).  This reflects transfer
        from the first stage subcortical pathway to the
        first stage cortical pathway.

    \item Lesions at the latest stages of learning -- from
        trial 1000 -- do not impair learning regardless of
        whether the lesion is to the DMS (Fig.
        \ref{fig_dms_1000}) or the DLS (Fig.
        \ref{fig_dls_1000}).  This reflects transfer from
        both subcortical pathways to both cortical pathways.

\end{itemize}

\begin{figure}[h!]
    \centering
    \includegraphics[width=0.7\textwidth]{../figures/fig_model.png}
    \caption{
        The model architecture. The model consists of two
        stages, each with a subcortical pathway (DMS and
        DLS) and a cortical pathway (VIS-PM and PM-M1).
        Dopamine-dependent, three-factor synaptic plasticity
        is implemented at all cortical-subcortical synapses
        (VIS-DMS and PM-DLS). Two-factor Hebbian learning
        governs all cortical-cortical synapses (VIS-PM and
        PM-M1).
    }
    \label{fig_model}
\end{figure}

\begin{figure}[h!]
    \centering
    \includegraphics[width=1.0\textwidth]{../figures/model_spiking_lesion_trials_100-1999_cells_DMS.png}
    \caption{
        Lesions to the DMS at early stages of learning --
        trial 100 onward -- prevent learning.
        %
        The first two columns show activity (i.e.,
        neurotransmitter release) across every time step and
        trial in every neuron.
        %
        The third column shows synaptic weights across
        trials.
        %
        The fourth column shows a variety of information
        pertaining the model's performance across trials.
        The first row of this column shows predicted rewards
        and RPE. The second row shows response accuracy (0
        for incorrect and 1 for correct). The row column
        shows response time.
    }
    \label{fig_dms_100}
\end{figure}

\begin{figure}[h!]
    \centering
    \includegraphics[width=1.0\textwidth]{../figures/model_spiking_lesion_trials_100-1999_cells_DLS.png}
    \caption{
        Lesions to the DLS at early stages of learning --
        trial 100 onward -- prevent learning.
        %
        The first two columns show activity (i.e.,
        neurotransmitter release) across every time step and
        trial in every neuron.
        %
        The third column shows synaptic weights across
        trials.
        %
        The fourth column shows a variety of information
        pertaining the model's performance across trials.
        The first row of this column shows predicted rewards
        and RPE. The second row shows response accuracy (0
        for incorrect and 1 for correct). The row column
        shows response time.
    }
    \label{fig_dls_100}
\end{figure}

\begin{figure}[h!]
    \centering
    \includegraphics[width=1.0\textwidth]{../figures/model_spiking_lesion_trials_500-1999_cells_DMS.png}
    \caption{
        Lesions to the DMS at intermediate stages of
        learning -- trial 500 onward -- do not impair
        learning, reflecting transfer from the first stage
        subcortical pathway to the first stage cortical
        pathway.
        %
        The first two columns show activity (i.e.,
        neurotransmitter release) across every time step and
        trial in every neuron.
        %
        The third column shows synaptic weights across
        trials.
        %
        The fourth column shows a variety of information
        pertaining the model's performance across trials.
        The first row of this column shows predicted rewards
        and RPE. The second row shows response accuracy (0
        for incorrect and 1 for correct). The row column
        shows response time.
    }
    \label{fig_dms_500}
\end{figure}

\begin{figure}[h!]
    \centering
    \includegraphics[width=1.0\textwidth]{../figures/model_spiking_lesion_trials_500-1999_cells_DLS.png}
    \caption{
        Lesions to the DLS at intermediate stages of
        learning -- trial 500 onward -- continue to impair
        learning, reflecting the lack of transfer from the
        second stage subcortical pathway to the second stage
        cortical pathway.  
        %
        The first two columns show activity (i.e.,
        neurotransmitter release) across every time step and
        trial in every neuron.
        %
        The third column shows synaptic weights across
        trials.
        %
        The fourth column shows a variety of information
        pertaining the model's performance across trials.
        The first row of this column shows predicted rewards
        and RPE. The second row shows response accuracy (0
        for incorrect and 1 for correct). The row column
        shows response time.
    }
    \label{fig_dls_500}
\end{figure}

\begin{figure}[h!]
    \centering
    \includegraphics[width=1.0\textwidth]{../figures/model_spiking_lesion_trials_1000-1999_cells_DMS.png}
    \caption{
        Lesions to the DMS at the latest stages of learning
        -- trial 1000 onward -- do not impair learning,
        reflecting transfer from the first stage subcortical
        pathway to the first stage cortical pathways. This
        is just a contunation of the results in Fig.
        \ref{fig_dms_500}.
        %
        The first two columns show activity (i.e.,
        neurotransmitter release) across every time step and
        trial in every neuron.
        %
        The third column shows synaptic weights across
        trials.
        %
        The fourth column shows a variety of information
        pertaining the model's performance across trials.
        The first row of this column shows predicted rewards
        and RPE. The second row shows response accuracy (0
        for incorrect and 1 for correct). The row column
        shows response time.
}
    \label{fig_dms_1000}
\end{figure}

\begin{figure}[h!]
    \centering
    \includegraphics[width=1.0\textwidth]{../figures/model_spiking_lesion_trials_1000-1999_cells_DLS.png}
    \caption{
        Lesions to the DLS at the latest stages of learning
        -- trial 1000 onward -- do not impair learning,
        reflecting transfer from the second stage
        subcortical pathway to the second stage cortical
        pathways.
        %
        The first two columns show activity (i.e.,
        neurotransmitter release) across every time step and
        trial in every neuron.
        %
        The third column shows synaptic weights across
        trials.
        %
        The fourth column shows a variety of information
        pertaining the model's performance across trials.
        The first row of this column shows predicted rewards
        and RPE. The second row shows response accuracy (0
        for incorrect and 1 for correct). The row column
        shows response time.
    }
    \label{fig_dls_1000}
\end{figure}

\end{document}

